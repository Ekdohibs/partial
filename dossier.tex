\documentclass[a4paper,11pt]{article}
\usepackage[margin=3cm]{geometry}
\usepackage[utf8]{inputenc}
\usepackage{listings, verbatim}
\usepackage{graphicx}
\usepackage{amsmath, amsthm, amssymb, textcomp, alltt, tcolorbox, framed}
\renewcommand{\thesection}{}
\renewcommand{\thesubsection}{}
\newcommand{\code}[1]{{\fontfamily{pcr}\selectfont #1}}

\lstset{basicstyle={\fontfamily{pcr}\selectfont}}
\begin{document}

\title{Évaluation partielle}
\date{}
\maketitle

\section{I. Présentation} 

\subsection{1. Définition d'un évaluateur partiel}

Le but d'un évaluateur partiel est d'optimiser un programme dont on
connaît une partie des arguments. Plus précisément, il s'agit d'une
fonction \code{specialize} vérifiant~: pour toute fonction \code{f},
pour toute entrée \code{x}, \code{y} de cette fonction,
\code{(specialize(f,~x))(y)~=~f(x,~y)}.
On souhaite naturellement que \code{(specialize(f,~x))(y)}
s'éxécute plus rapidement que \code{f(x,~y)}, en profitant du fait que
\code{x} est déjà connu au calcul de \code{specialize(f,~x)}.
\subsection{2. Applications}

L'évaluation partielle a de nombreuses applications, dont on pourrait
par exemple citer (c.f. [1] p.xi)~:

\begin{itemize}

\item Génération de code et compilation d'un langage en disposant
uniquement d'un interpréteur de celui-ci,

\item Entrainement de réseaux de neurones~: un évaluateur partiel permet
de transformer un programme général en un adapté pour entrainer un
réseau donné, et qui sera donc plus rapide,

\item Simulation numérique de circuits.

\end{itemize}

Je me suis plus particulièrement intéressé au premier point, pour
plusieurs raisons~: en plus de son intérêt pratique (un interpréteur
est bien plus simple à écrire qu'un compilateur), il permet en effet
de faire des tests de performance de l'évaluateur partiel en comparant
le résultat au programme qui serait naturellement écrit dans le
langage cible ou bien compilé à partir du programme initial.

\section{II. Conception d'un langage simple}

\subsection{1. Choix de conception}

Pour pouvoir écrire un évaluateur partiel, il me fallait commencer par
un interpréteur. J'ai donc créé un langage assez simple, dont la
syntaxe est assez proche du Lisp (par souci de simplicité
d'implémentation). Les langages impératifs conduisant à des effets de
bord, ils ne sont pas souhaitables pour l'écriture d'un évaluateur
partiel, car les effets de bord conduisent à une complexité accrue, en
particulier lors des phases d'optimisation. Au contraire, un langage
purement fonctionnel vérifie des propriétés comme le fait que résultat
ne dépend pas de l'ordre d'évaluation ou qu'une fonction peut être
mémoïzée, qui rendent les optimisations plus simples et dont la
correction est plus simple à montrer.  Un autre choix important est la
portée des variables~: ayant commencé par utiliser une portée
dynamique, pour faciliter l'implémentation, je me suis vite aperçu de
mon erreur (c.f. note 1) \\

\begin{tcolorbox}
{\Large Note 1 \normalsize(Un problème avec la portée dynamique)}

\begin{lstlisting}[frame=single]
(let m 1)
(let (f x)
     (+ x m))
(let (g m)
     (f m))
\end{lstlisting}

Ce code possède un problème majeur~:
Dans le calcul de \code{(g 2)}, on appelle \code{(f 2)}.
Ce calcul cherche la valeur de \code{m}; or
celle-ci était 1 lors de la définition de
\code{f}, mais quand on appelle, \code{(g 2)}, la valeur de \code{m}
est 2 dans l'environnement de \code{g}, puis dans celui
de \code{f} \emph{à cause de la portée dynamique}. On obtient
donc le résultat 4 au lieu de 3 attendu : en particulier,
le résultat de \code{(f 2)} \emph{dépend de l'environnement} dans
lequel cet expression est calculée.
\end{tcolorbox}

Après avoir choisi d'utiliser une portée statique, un autre problème
se présentait~: celui des fonctions récursives. En effet, avec la
portée dynamique, une fonction est dans son propre environnement au
moment de son appel; mais avec la portée statique, comme
l'environnement d'une fonction est réduit à celui qui existait au
moment de sa définition plus ses arguments, elle n'est pas dans son
propre environnement~! J'avais alors deux possibilités à ma
disposition~: la première était d'utiliser un combinateur de point
fixe, la deuxième étant d'utiliser un environnement circulaire. J'ai
choisi la première possibilité, bien que pour des raisons de confort
(aussi bien au niveau de l'écriture d'un programme que de l'écriture
de l'interpréteur), ce combinateur de point fixe soit implémenté dans
l'interpréteur lui-même (c.f. note 2). En effet, un environnement
circulaire possède le problème qu'il nécessite de créer une structure
de données circulaire (ce qui n'est par exemple pas possible dans le
langage lui-même), qui peut ensuite poser des problèmes (lors d'un
test d'égalité structurelle par exemple). \\

\begin{tcolorbox}
{\Large Note 2 \normalsize (Le combinateur de point fixe et son utilisation)} \\
  Un combinateur de point fixe \code{Y} vérifie : pour tout \code{f},
  \code{Y~f~=~f~(Y~f)}.  Il permet de faire des récursions en
  transformant une expression du type~:
\begin{lstlisting}
(letrec (f x) <definition de f>)
\end{lstlisting}
en:
\begin{lstlisting}
(let (f_rec f x) <definition de f>)
(let f (Y f_rec))
\end{lstlisting}
Le combinateur de point fixe est implémenté ici directement dans
l'interpréteur, ainsi \code{(Y f\_rec)} retourne une valeur de type
point fixe, qui ne sera développé que lorsque l'on essaiera de
l'appliquer à des arguments.  Une version légèrement généralisée du
combinateur de point fixe, qui prend plusieurs arguments est utilisée
pour les fonctions mutuellement récursives.
\end{tcolorbox}

\subsection{2. L'interpréteur : la boucle évaluer / appeler}

L'interpréteur se base sur deux fonctions mutuellement récursives,
"évaluer" et "appeler". "évaluer" prend en entrée une expression et un
environnement, tandis que "appeler" prend en entrée une fonction et
ses arguments, et calcule le résultat de l'application de cette
fonction à ces arguments (elle prend donc des valeurs et non des
expressions en entrée).

\begin{comment}
Le code présenté ici est légèrement simplifié,
afin de montrer clairement sa structure de base.

\begin{lstlisting}
evaluer(expression, environnement)
  Si expression est une constante:
    retourner expression
  Si expression est une variable:
    retourner trouver_variable(environnement, expression)
  Si expression est une instruction conditionnelle:
    resultat_test = evaluer(test(expression), environnement)
    Si resultat_test:
      retourner evaluer(si_vrai(expression), environnement)
    Sinon:
      retourner evaluer(si_faux(expression), environnement)
  Si expression est un appel:
    fct = evaluer(fonction(expression), environnement)
    arguments = [evaluer(arg, environnement)
        pour arg dans arguments(expression)]
    appeler(fct, arguments)

appeler(fonction, arguments)
  Si fonction est une fonction de base:
    appeler_fonction_de_base(fonction, arguments)
  Sinon:
    env = environnement_fonction(fonction)
    nouveau_env = ajout_environnement(env,
        noms_arguments(fonction), arguments)
    retourner evaluer(code(fonction), nouveau_env)
\end{lstlisting}
\end{comment}

\section{III. L'évaluateur partiel}

\subsection{1. Modification de l'interpréteur en évaluateur partiel}

On remarque que un interpréteur et un évaluateur partiel sont en fait
très proches : on va donc modifier l'interpréteur. La première
modification est le type des valeurs : on ajoute un type spécial
\code{<inconnu>} qui signifie qu'on peut remplacer cette valeur par n'importe
quelle autre.

\begin{framed}
{\noindent \Large Définition 1} \\
\indent On dit qu'une valeur $v$ est \emph{compatible} avec une
valeur $\tilde{v}$ utilisant des \code{<inconnu>} si pour chaque
occurence de \code{<inconnu>} dans $\tilde{v}$ il existe une valeur
de manière à ce que remplacer chaque occurence de \code{<inconnu>}
dans $\tilde{v}$ par la valeur associée donne $v$.

Par exemple, \code{(cons~1~2)} est compatible avec \code{(cons~1~<inconnu>)}.
\end{framed}

Uniquement modifier l'interpréteur pour qu'il accepte et retourne des
valeurs de ce type étendu n'est cependant pas suffisant: on obtient
bien une valeur du type étendu qui correspond à la valeur de retour de
l'expression, mais on ne sait pas comment obtenir ce résultat~! On
modifie donc également la valeur de retour de l'interpréteur pour
qu'il retourne également une expression calculant la valeur, et tenant
compte des restrictions imposées aux arguments.

\begin{comment}
On obtient donc le pseudo-code suivant, très proche de celui de
l'interpréteur (il est ici écourté) :

\begin{lstlisting}
specialiser(expression, environnement)
  Si expression est une constante:
    retourner (expression, expression)
  Si expression est une variable:
    retourner (trouver_variable(environnement, expression),
                   expression)
  Si expression est une instruction conditionnelle:
    (resultat_test, code_test) =
        specialiser(test(expression), environnement)
    Si resultat_test est connu:
        Si resultat_test:
          retourner specialiser(si_vrai(expression),
                                environnement)
        Sinon:
          retourner specialiser(si_faux(expression),
                                environnement)
    Sinon:
       (valeur1, code1) =
           specialiser(si_vrai(expression), environnement)
       (valeur2, code2) =
           specialiser(si_faux(expression), environnement)
       retourner (fusion_valeurs(valeur1, valeur2),
                    creer_test(code_test, code1, code2))
  etc.
\end{lstlisting}
\end{comment}

Une fois le calcul de l'évaluateur partiel terminé, on effectue une
phase d'optimisation~: en effet, on se retrouve avec beaucoup plus de
fonctions que l'on en avait au départ, mais de nombreuses fonctions
sont triviales et se réduisent à l'appel d'une autre fonction, ou
n'utilisent pas certains de leurs arguments. On simplifie donc cela,
puis on recrée le code permettant de calculer le résultat en mettant
bout-à-bout les définitions des différentes fonctions (après calcul
des composantes fortement connexes du graphe d'appel des fonctions
pour savoir quelles définitions récursives sont nécessaires).

\subsection{2. Vérification de la validité}

On suppose ici que le calcul de l'évaluateur partiel termine (en
pratique, on ajoute des limites au nombre de fois où une fonction peut
être spécialisée pour assurer cela). On peut alors montrer le résultat
suivant~:

\begin{framed}
{\noindent \Large Théorème 1} \\
\indent Pour tout appel de \code{specialize(expr,~env)} \emph{lors
de l'éxécution de l'évaluateur partiel sur un programme donné},
si on a \code{(v,~code)~=~specialize(expr,~env)} alors tout appel de
\code{eval(expr,~e)} \emph{qui termine} où \code{e} est un
environnement compatible avec \code{env}
(i.e. chaque valeur de \code{e} est compatible
avec la valeur correspondante de \code{env}) vérifiera :
\code{eval(expr,~e)} est compatible avec \code{v}, et de plus,
\code{eval(expr,~e)~=~eval(code, e)}.
\end{framed}
En particulier, il se peut qu'un programme qui ne termine pas soit
transformé en programme qui termine (c.f. note 3). On ne montre \emph{pas}
ce résultat
par induction structurelle (car l'appel d'une fonction ne
correspond pas à un cas d'induction structurelle) mais par récurrence
par ordre croissant de fin d'appel (c'est pour cela que l'on a supposé
que le calcul se terminait).

%Voir encadré pour la preuve d'un des cas.

\begin{tcolorbox}
{\Large Note 3 \normalsize (Programme dont la terminaison n'est pas respectée)}
\begin{lstlisting}[frame=single]
(letrec (f x) (f x))
(let (g x) (tail (cons (f x) 0)))
\end{lstlisting}
Le calcul de \code{(g 1)} ne termine pas, mais après évaluation partielle,
on obtient :
\begin{lstlisting}[frame=single]
(let (g x) 0)
\end{lstlisting}
qui termine. On aurait cependant obtenu le même résultat si le
programme initial terminait, par exemple si il était évalué
paresseusement.
\end{tcolorbox}

\begin{comment}
\begin{tcolorbox}
{\large Preuve 1: Correction dans le cas d'un \code{if}} \\
\Huge TODO (?)
\end{tcolorbox}
\end{comment}

\section{IV. Résultats}

Pour tester l'efficacité de l'évaluateur partiel, j'ai écrit un
meta-évaluateur du langage, et je l'ai évalué partiellement en lui
fixant un programme. J'ai utilisé la fonction naïve calculant la suite
de Fibonacci comme programme afin d'obtenir de nombreux appels de
fonctions pour des entrées assez petites: en effet, pour avoir des
temps mesurables, il fallait que les temps d'éxécution soient très
grands. Ceux-ci restent cependant exponentiels après évaluation
partielle~: celle-ci ne remplace pas un mauvais algorithme.

J'ai également comparé quatre versions différentes :

\begin{itemize}

\item Le metaévaluateur calculant la suite de Fibonacci [Metaeval]
\item Le même programme évalué partiellement [Partiel]
\item Le même programme évalué partiellement, mais sans la phase
d'optimisations [Partiel non opt]
\item La programme calculant la suite de Fibonacci qui était donné comme
entrée au metaévaluateur. [Simple]

\end{itemize}

De plus, j'ai comparé deux méthodes d'exécution :

\begin{itemize}
\item L'évaluateur initial,
\item Les programmes, compilés en OCaml puis vers du code natif.
\end{itemize}

\newpage
\noindent Résultats : \\ \\
Interprété~: \\
\includegraphics[width=13cm]{eval_graph} \\
Compilé~: \\
\includegraphics[width=13cm]{compiled_graph}
 \\

Pour le calcul de \code{(fib 20)} : \\

\begin{tabular}{ | l | c | c | c | c | } \hline
       	& Metaeval & Partiel & Partiel non opt & Simple \\ \hline
Évalué 	& 211.2 s  & 1.206 s & 39.03 s         & 0.813 s \\ \hline
Compilé & 0.304 s  & 0.005 s & 0.015 s         & 0.003 s \\ \hline

\end{tabular}
\\

On remarque en particulier que le temps d'exécution du programme
partiellement évalué est proche de celui du programme donné en
argument au metaévaluateur, ce qui était bien le résultat
souhaité. On voit également que la phase d'optimisation, réduit de
manière importante le temps d'éxécution. Sans celle-ci, le programme
partiellement évalué reste assez lent mais est quand même bien plus
rapide que le metaévaluateur.  C'est aussi en supprimant cette phase
d'optimisation qu'on trouve la seule différence significative de temps
d'éxécution entre la version compilée et interprétée (le compilateur
de OCaml faisant quelques optimisations).

\section{V. Améliorations envisageables}

Pour le cœur de l'évaluateur partiel, on pourrait envisager quelques
améliorations par exemple lors de la spécialisation de fonctions
récursives : en effet une fonction du type :

\begin{lstlisting}
(letrec (f n) (if (= n 0) 0 (f (- n 1))))
\end{lstlisting}

qui retourne toujours 0 quand elle termine n'est pas détectée en tant
que telle.

Une autre amélioration, cette fois-ci lors de l'optimisation
pourrait être de simplifier les fonctions qui retournent une structure
partiellement connue à l'évaluation (du type \code{(cons~0~<inconnu>)}). La
valeur de retour de telles fonctions n'est en effet pas modifiée lors
de l'optimisation. Une autre manière d'aborder ce problème serait de
compiler le langage vers un langage fonctionnel doté d'un bon compilateur
optimisant (possiblement Haskell), et de laisser ce compilateur faire
toute la phase d'optimisation une fois la phase d'évaluation partielle faite
(qui resterait utile même si le langage était compilé, le compilateur
n'effectuant pas d'évaluation partielle).

\section{VI. Bibliographie}

\begin{description}

\item[{[1]}] Jones, Neil D.; Gomard, Carsten K.; Sestoft, Peter. Partial Evaluation and Automatic Program Generation. 1993, 425 p.
\item[{[2]}] Futamura, Yoshihiko, Partial Evaluation of Computation Process — An Approach to a Compiler-Compiler. Higher-Order and Symbolic Computation 1999, n°12, pp. 381-391
\item[{[3]}] Abelson, Hal; Sussman, Jerry; Sussman, Julie. Structure and Interpretation of Computer Programs. 2e éd. MIT Press, 1996, pp. 362-397.

\end{description}

\end{document}