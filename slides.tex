\documentclass[mathserif]{beamer}
\usepackage[utf8]{inputenc}
\usepackage{listings, verbatim}
\usepackage{graphicx}
\usepackage{amsmath, amsthm, amssymb, textcomp, alltt, tcolorbox, framed}
%\renewcommand{\thesection}{}
%\renewcommand{\thesubsection}{}
\newcommand{\code}[1]{{\fontfamily{pcr}\selectfont #1}}
%\usetheme{Madrid}
\lstset{basicstyle={\fontfamily{pcr}\selectfont}}
\begin{document}

\title{Évaluation partielle}
\date{}
\frame{\titlepage}

\begin{frame}
\frametitle{Sommaire}
\tableofcontents
\end{frame}

\section{Introduction} 

%\subsection{1. Présentation}

\begin{frame}
\frametitle{Présentation}

On donne~:

\begin{itemize}

\item Le code d'une fonction \code{f}

\item Un argument \code{x} de \code{f}

\end{itemize}

But~: Créer une fonction \code{g} telle que~:


\begin{itemize}

\item $\forall$\code{y}, \code{f(x,~y)~=~g(y)}

\item Le calcul de \code{g(y)} est plus rapide que celui
de \code{f(x,~y)}

\end{itemize}

\end{frame}

%\subsection{2. Applications}

\begin{frame}
\frametitle{Applications}

De nombreuses applications existent, dont~:

\begin{itemize}

\item Génération de code et compilation d'un langage en disposant
uniquement d'un interpréteur de celui-ci,

\item Entrainement de réseaux de neurones,

\item Simulation numérique de circuits.

\end{itemize}

\end{frame}

\section{Conception d'un langage simple}

\begin{frame}
\frametitle{Pourquoi créer un nouveau langage~?}

\begin{itemize}
\item Disposer d'un langage minimaliste
\item Disposer d'un interpréteur pour celui-ci
\end{itemize}

\end{frame}

%\subsection{1. Choix de conception}

\begin{frame}
\frametitle{Choix de conception}

\begin{itemize}

\item Syntaxe~: proche du Lisp

\item Typage dynamique

\item Portée statique

\item Utilisation d'un combinateur de point fixe pour la récursion

\end{itemize}

\end{frame}

%\subsection{2. L'interpréteur : la boucle évaluer / appeler}

\begin{frame}
\frametitle{L'interpréteur - la boucle évaluer / appeler}

\begin{itemize}
\item Basé sur deux fonctions mutuellement récursives : \code{évaluer}
et \code{appeler}

\item \code{évaluer(expr,~env)} retourne le résultat de l'évaluation de
\code{expr} dans l'environnement \code{env}

\item \code{appeler(fct,~args)} retourne le résultat de l'appel de \code{fct}\\
avec les arguments \code{args}

\end{itemize}

\end{frame}

\section{L'évaluateur partiel}

%\subsection{1. Modification de l'interpréteur en évaluateur partiel}
\begin{frame}
\frametitle{Modification de l'interpréteur}

On ajoute une valeur spéciale \code{<inconnu>} et on définit~:

\begin{framed}
{\Large Définition} \\
On dit qu'une valeur $v$ est \emph{compatible} avec une
valeur $\tilde{v}$ utilisant des \code{<inconnu>} si pour chaque
occurence de \code{<inconnu>} dans $\tilde{v}$ il existe une valeur
de manière à ce que remplacer chaque occurence de \code{<inconnu>}
dans $\tilde{v}$ par la valeur associée donne $v$.
\end{framed}

Par exemple, \code{(cons~1~(cons~2~3))} est compatible avec
\code{(cons~<inconnu>~(cons~2~<inconnu>))}.

\end{frame}

\begin{frame}
\frametitle{Modification de l'interpréteur}

On modifie ensuite l'interpréteur en une fonction \code{specialize}
pour que la propriété suivante soit vérifiée~:

\begin{framed}
{\Large Propriété} \\
Pour tout appel de \code{specialize(expr,~env)} \emph{lors
de l'éxécution de l'évaluateur partiel sur un programme donné},
si on a \code{(v,~code)~=~specialize(expr,~env)} alors tout appel de
\code{eval(expr,~e)} \emph{qui termine} où \code{e} est un
environnement compatible avec \code{env}
(i.e. chaque valeur de \code{e} est compatible
avec la valeur correspondante de \code{env}) vérifiera :
\code{eval(expr,~e)} est compatible avec \code{v}, et de plus,
\code{eval(expr,~e)~=~eval(code, e)}.
\end{framed}

\end{frame}

\section{Résultats}

\begin{frame}
\frametitle{Méthode d'expérimentation}

On compare les programmes suivants~:
\begin{itemize}
\item Un programme calculant la suite de Fibonacci naïvement
\item Le metaévaluateur exécutant ce programme
\item Ce dernier évalué partiellement, avec et sans optimisations
\end{itemize}
On compare également deux méthodes d'éxécution différentes~:
\begin{itemize}
\item L'évaluateur (écrit en Python) qui éxécute ces programmes,
\item Les programmes, compilés vers OCaml puis vers du code natif.
\end{itemize}

\end{frame}

\begin{frame}
\frametitle{Temps d'éxécution}
\includegraphics[width=5.5cm,height=4cm]{eval_graph}
\includegraphics[width=5.5cm,height=4cm]{compiled_graph}
\end{frame}

\section{Améliorations envisageables}

\begin{frame}[fragile]
\frametitle{Améliorations envisageables}

\begin{itemize}
\item Gérer les fonctions comme~:
\begin{lstlisting}
(letrec (f n) (if (= n 0) 0 (f (- n 1))))
\end{lstlisting}
\item Améliorer la phase d'optimisation ou utiliser un compilateur
optimisant déjà existant
\end{itemize}
\end{frame}

\section{Bibliographie}

\begin{frame}
\frametitle{Bibliographie}
\begin{description}

\item[{[1]}] Jones, Neil D.; Gomard, Carsten K.; Sestoft, Peter. Partial Evaluation and Automatic Program Generation. 1993, 425 p.
\item[{[2]}] Futamura, Yoshihiko, Partial Evaluation of Computation Process — An Approach to a Compiler-Compiler. Higher-Order and Symbolic Computation 1999, n°12, pp. 381-391
\item[{[3]}] Abelson, Hal; Sussman, Jerry; Sussman, Julie. Structure and Interpretation of Computer Programs. 2e éd. MIT Press, 1996, pp. 362-397.

\end{description}
\end{frame}
\end{document}