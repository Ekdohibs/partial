\documentclass[mathserif]{beamer}
\usepackage[utf8]{inputenc}
\usepackage{listings, verbatim}
\usepackage{graphicx}
\usepackage{amsmath, amsthm, amssymb, textcomp, alltt, tcolorbox, framed}
%\renewcommand{\thesection}{}
%\renewcommand{\thesubsection}{}
\newcommand{\code}[1]{{\fontfamily{pcr}\selectfont #1}}
%\usetheme{Madrid}
\lstset{basicstyle={\fontfamily{pcr}\selectfont}}
\lstdefinelanguage{L}
  {morekeywords={head,cons,tail,if,let,letrec,let\*,letrec\*},
   sensitive,
   alsodigit=-,
   alsoletter={.:,*=-+[]},
   morecomment=[l];,
   morestring=[b]"
  }[keywords,comments,strings]

\begin{document}

\begin{frame}
\frametitle{Temps d'éxécution}
\includegraphics[width=5.5cm,height=4cm]{eval_graph}
\includegraphics[width=5.5cm,height=4cm]{compiled_graph}
\end{frame}


\begin{frame}[fragile]
\frametitle{L'interpréteur : évaluer}
\begin{lstlisting}
evaluer(expr, env)
  Si expr est une constante:
    retourner expr
  Si expr est une variable:
    retourner trouver_variable(env, expr)
  Si expr est une instruction conditionnelle:
    resultat_test = evaluer(test(expr), env)
    Si resultat_test:
      retourner evaluer(si_vrai(expr), env)
    Sinon:
      retourner evaluer(si_faux(expr), env)
  Si expr est un appel:
    fct = evaluer(fonction(expr), env)
    arguments = [evaluer(arg, env)
        pour arg dans arguments(expr)]
    appeler(fct, arguments)
\end{lstlisting}
\end{frame}
\begin{frame}[fragile]
\frametitle{L'interpréteur : appeler}
\begin{lstlisting}
appeler(fonction, arguments)
  Si fonction est une fonction de base:
    appeler_fonction_de_base(fonction, arguments)
  Sinon:
    env = environnement_fonction(fonction)
    nouveau_env = ajout_environnement(env,
        noms_arguments(fonction), arguments)
    retourner evaluer(code(fonction), nouveau_env)
\end{lstlisting}
\end{frame}

\begin{frame}[fragile]
\frametitle{L'évaluateur partiel}
{\small
\begin{lstlisting}
specialiser(expr, env)
 [...]
 Si expr est une instruction conditionnelle:
   (resultat_test, code_test) =
       specialiser(test(expr), env)
   Si resultat_test est connu:
       Si resultat_test:
         retourner specialiser(si_vrai(expr), env)
       Sinon:
         retourner specialiser(si_faux(expr), env)
   Sinon:
      (valeur1, code1) = specialiser(si_vrai(expr), env)
      (valeur2, code2) = specialiser(si_faux(expr), env)
      retourner (fusion_valeurs(valeur1, valeur2),
                   creer_test(code_test, code1, code2))
 [...]
\end{lstlisting}
}
\end{frame}

\begin{frame}[fragile]
\frametitle{Exemple d'évaluation partielle}
Entrée~:
\begin{lstlisting}[language=L]
(letrec (rev_concat l r)
    (if (= l [])
        r
        (rev_concat
            (tail l)
            (cons (head l) r))))
(rev_concat [a b c d] [])
\end{lstlisting}
Sortie~:
\begin{lstlisting}
[d c b a]
\end{lstlisting}
\end{frame}

\begin{frame}[fragile]
\frametitle{Résultats de l'optimisation}
Sans optimisation~: \\
{\tiny
\begin{lstlisting}[language=L]
(letrec* (
((eval_expr-6 env-5 expr-4) (let (cons type-2 ex-2) globalvar-12
  (eval_keyword-1 env-5 globalvar-13)))
((eval_call-1 func-3 args-4) (let (cons code-1 (cons argnames-1 (cons fenv-1 [])))
  globalvar-23 (eval_expr-6 (add_list_to_env-1 globalvar-7 globalvar-24 args-4)
    globalvar-12)))
((eval_expr-10 env-12 expr-10) (let (cons type-6 ex-6) globalvar-28 (let (cons ft-4 func-5)
  globalvar-29 (let (cons ft2-1 func2-1) globalvar-30 (eval_call-1 globalvar-23
    (eval_args-5 env-12 globalvar-31))))))
((eval_expr-12 env-18 expr-13) (let (cons type-7 ex-7) globalvar-37 (let (cons ft-5 func-6)
  globalvar-29 (let (cons ft2-2 func2-2) globalvar-30 (eval_call-1 globalvar-23
    (eval_args-7 env-18 globalvar-38))))))
((eval_args-9 env-27 args-12) (cons (eval_expr-12 env-27 globalvar-37) globalvar-17))
((eval_args-4 env-90 args-42) (cons (eval_expr-10 env-90 globalvar-28)
  (eval_args-9 env-90 globalvar-46)))
((eval_expr-5 env-11 expr-9) (let (cons type-5 ex-5) globalvar-11 (let (cons ft-3 func-4)
  globalvar-1 (let args-5 (eval_args-4 env-11 globalvar-27) (tagged_+-1 args-5)))))
((eval_keyword-1 env-4 expr-3) (let keyword-1 globalvar-8 (let (cons ct-1 condition-1)
  (eval_expr-3 env-4 globalvar-9) (if condition-1 (eval_expr-4 env-4 globalvar-10)
    (eval_expr-5 env-4 globalvar-11)))))
))
\end{lstlisting}
}
%\includegraphics[width=11cm]{without_opt} \\
Avec optimisation~: \\
{\tiny
\begin{lstlisting}[language=L]
(letrec (eval_keyword-1 env-465-h-t-t-1) (if (<= env-465-h-t-t-1 1) (cons 0 env-465-h-t-t-1)
  (cons 0 (+ (tail (eval_keyword-1 (- env-465-h-t-t-1 2)))
             (tail (eval_keyword-1 (- env-465-h-t-t-1 1)))))))
\end{lstlisting}
}
\end{frame}

\end{document}